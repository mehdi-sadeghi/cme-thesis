\chapter{Rough Ideas}

This chapter contains very raw ideas to address main requirements i.e.
distributed workflow management and intelligent data transfer.

\section{Intelligent Data Transfer: Use Case One}
By \textit{Intelligent Data Transfer} we mean an approach that
minimizes required data transfers between application\footnote{To be defined}
instances.

In the most basic use case\footnote{To be added later and 
referenced here} we run a script
\footnote{To be defined and added to
the terminology, terminology itself has to be defined}
which consists of two linear operations. Each operation consumes data
and generates data. A third operation needs both generated data two 
operate on and generate the third and final data.

The script is data driven. It means that it contains a number
of steps and for every step it needs appropriate data to run the
desired operations\footnote{To be defined}. We assume that
the script will run on \textit{Node 1} and required data 
\textit{DataSet1} and \textit{DataSet2} are 
located on \textit{Node 2}
and \textit{Node 3} respectively. Therefor \textit{Node 1} have
to initiate operations on the other machines.

\subsection{Identical Instances}
We assume that on each machine of the network the same instance of our
imaginary program is running which is capable of running all operations
including A, B and C. The only consideration is the availability of 
DatSets, they are not available on all machines.

A linear operation (not clear to my self how to write it):

\[ Operation(A, B) = Operation(A) + Operation(B) \]

\[ DataSet^A = Operation^A(DataSet^1) \]

\[ DataSet^B = Operation^B(DataSet^2) \]

\[ DataSet^C = Operation^C(DataSet^A, DataSet^B) \]

Assuming that operations A and B will run on the machines which
contain the required data, a number of questions arise here:
\begin{enumerate}
\item On which machine operations C should run? A, B or C?
\item On How to transfer the required data to that machine in an 
optimized way?
\end{enumerate}

\subsection{The Idea}
First of all we assume that we have the information about the DataSets
available on all of the mamchines i.e. in form of a distributed table
with entries containing the node address and DataSet id. Based on this
information the application can decide if it has the required data or
not. 

Based on this algorithm (to be defined) the initial application
delegates operations to the other nodes (instances of the same program),
where the data is available. Our distributed workflow manager (to be 
defined) will synchronize the information on these running operation and
will lable the output data and will add it to the distributed data table.

After finishing operations A and B we will run operaion C in either
of these nodes, because the required data is partially available on these
nodes. Then we have to transfer the rest of the data to one of these
nodes to run the operation C which needs both parts simultaniously.

\subparagraph{Using Prior Art}
At this point we can take advantage of existing Distributed File Systems
(DFS) to make the data available for operation C. We can then eliminate
the complexity of data transfer between these two nodes and delegate it
to existing distributed file systems. The main point is we don't rely on
DFS for all of our decistion making part but we explicitely make the 
decition to run operation A and B on specific nodes and then for the 
last part we use a meta disk or universal disk concept to deliver the
remaining data for operation C.