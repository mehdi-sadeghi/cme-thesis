\chapter{Requirements}
\label{cha:requirements}

Every group has its own needs. That is the reason that we have many different solutions in the market.
In this chapter we explain the requirements according to our context.

\section{Data Location Abstraction}
We need to abstract the absolute path of required data from users. To run every operation
we provide certain data files as input. This is part of the manual step of running an operation and
it makes it fragile. Currently users have to take care of storing input files in correct folders before
running their scripts, or they have to copy the input files from the shared network file system to the
appropriate runtime folder. This is something that we want to disappear. We want the system to 
manage the input data and its absolute location.

\section{Server Agnostic}
We want to have the same user experience regardless of the machine that we connect to. It is very common
that a user moves from one workstation to another one or connects to different machines using SSH. We 
want to be able to provide the requested information to the user, no matter to which machine she connects.

We could have any number of active computers in our network which are running an instance of our program
and we want to let the users to talk to each of them and be able to launch same operations and get the same result.

It also means that if one users initiates an operation in first computer and then goes to the next one and asks
for the status of the operation, she should be able to do that as if she is working with the first computer.

\section{Distributed Solution}
We need a distributed solution which eliminates the need to have central orchestration.
We want to decrease the dependency between running programs therefore we want to have 
some sort of distributed application which distributes the operations to all other computers.

\section{Distributed Data}
We want to be able to store the result of an operation on different computers, i.e. distribute it on the network.
This comes from the nature of our workflows. Normally we have huge datasets which represent certain models i.e.
particles of a fluid or gas inside a container and we want to run multiple operations on those datasets. These
are normally available on different computers of an institute, and if there are multiple institutes cooperating
on a topic then we have to fetch data from remote computers. Therefore we need to consider distributed data management.

\section{Runtime Control}
While an operation is running we want to have full control over every step. In traditional approaches using job
scheduling systems this is not possible. Except basic control such as stop, resume or similar operations, users
can not control the runtime behavior of their program. In contrast we want to control all aspects of an operation.

\section{Easy Deployment}
We look for a solution that is easy to deploy onto new machines. A fully automated installation process
is required. Unnecessary dependencies should be avoided. Users in science field are not professionals in the field
of computer and therefore installing complex software requires system administrators to come in. Such installations
costs money because user herself is not able to finish it without help from others. Moreover it makes it non-feasible
for single users or small groups to try the software.

\section{User Space Solution}
We need a user space solution. It means that it should be possible to deploy, install and run the software without
having full control over the machine which is supposed to run it. Any software which needs administrator or root access
rights is not of our concern. Therefore the software itself along all of its dependencies should be installed and should
be able to work correctly in user space, under a normal Windows or Linux user account with no special rights except the
default ones.

\section{OS Agnostic}
We want to be able to run the solution on both Linux machines and Windows machines. It is common that users have
two machines, one for office tasks with Windows OS and the other for running simulations with Linux. However Linux
is the favorite machine to run the software but it could be possible to run it on Windows as well.

\section{Light Weight}
We look for a light weight solution which could be run on both laptops and stronger machines, with minimum dependencies
and launch time.


\section{Assumptions}
During this work we have a number of assumptions. We have a certain problem which we want to focus
on rather than reintroducing solutions that already exist. Here are the assumptions that we have
considered during this work:

%\subsection{Collaborating Network}
%We assume there is a network of computers which are available to run the tasks, each node is running an instance of the application. We will propose our collaboration and data transfer algorithm between them later.

\subsection{Data Characteristics}
We need to discuss more about the data. In our scientific context data is mostly numerical and explains characteristics
of physical particles such as atoms and molecules. These data is being used to simulate collections of particles called
models. Although our work is not dependent on these, they help us to understand the definition of the data that
we often refer to in this report. One important aspect of the data that we are interested in is that it is not critical 
and we can reproduce it. 

\subsection{Data Transfer}
We assume a data transfer approach is already in place. This could be any file system which supports 
network storage. Rather than going into details of how data could be transferred more efficiently, we will
focus on finding which data to be transferred and from which computer to which destination.

%\subsection{Workflow}
%In contrast to data we are interested in workflow. We want to find a reliable approach to access and update state of our workflow on any arbitrary node which is part of our collaborative network.
