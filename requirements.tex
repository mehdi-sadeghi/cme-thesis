\chapter{Requirements}
\label{cha:requirements}

Every group has its own needs. 
This is the reason that we have many different solutions in the market.
We focus on requirements of material science community in this work, 
even though the results could be applied to similar fields of science as well. 

One of the efforts in material science community is an ongoing European project called
 SimPhoNy\footnote{Simulation framework for multi-scale phenomena in micro- and nanosystems}.
It is focused on delivering a product which will run multiple distributed third-party simulation engines
under the cover of one application.
This application has to keep state of ongoing simulations and provide them with necessary datasets while 
running\footnote{\url{http://www.simphony-project.eu/about-simphony/}}.

Our requirements which will be introduced further in this chapter originate from the same context as in this project.

\section{Assumptions}
We have a certain problem which we want to focus on rather than reintroducing solutions that already exist. 
Here we go through the assumptions which are necessary to understand the next chapters.

\subsection{Data characteristics}
In our context, data is mostly numerical and explains characteristics of physical particles such as atoms and molecules. 
This data is being used in multiscale simulations. 
One important aspect of our datasets is that they are not critical and we can reproduce them. 
We are more concerned about preserving the running state of a distributed application rather than datasets, 
hence the focus on distributed workflow.

\subsection{Data transfer}
We assume a data transfer approach is already in place.
This could be any file system which supports 
network storage. Rather than going into details of how data could be transferred more efficiently, 
we will focus on finding which data to be transferred from a source to a destination computer.

\subsection{Independent work}
Our work is independent from any simulation or any context related tools.
We do not rely on any specific software or datasets 
and we tend to produce an abstract and general solution.
When we refer to domain related topics, 
it is to help us to better understand the practical use cases of this thesis.


\section{Requirements}
Before bringing in the requirements we take a moment to better understand the environment that these requirements come from.

It is common that users work with scientific data with help from SciPy 
project\footnote{"A Python-based ecosystem of open-source software for mathematics, science, and engineering" \url{www.scipy.org}}.
This project alone is a reason why we are biased toward Python, as later we will discuss.
SciPy project includes NumPy which is an N-dimentional array package which often is used by users and programs
to deal with datasets. 

There is a well known portable scientific data file format called 
HDF5\footnote{Hierarchical Data Format \url{http://www.hdfgroup.org/HDF5/}}.
This format is de facto standard for storing large datasets in many scientific fields which need to store numerical data.
Later on we will see that we have used HDF5 as storage in our prototype, even though we are not dependant on it.
NumPy and HDF5 play well together, i.e. there are many tools and libraries to store and retrieve NumPy data structurs 
within a HDF5 file. HDF5 does not need a runtime, having HDF5 library is enough to access and manipulate it.

It is also good to mention that our intended users normally work in places where there is a computer cluster
and they can access it using their own workstations via local area network (LAN).
It is also very common that they have minimum access rights to their workstations, 
since administrators want to prevent installation of programs that might increase the risk of problems and the cost of maintenance.

And one more point about operations that we will frequently talk about them in this text.
Users can run simulations on their own laptop and workstations (which is very common),
but when it comes to more time consuming tasks, they need to work with clusters.
Usually they login to remote machine or they use some local programs\footnote{Such as MoleQueue \url{http://www.openchemistry.org/projects/molequeue/}}
 that will submit their scripts to remote resources.
However this is not a versatile approach and therefore we need to change this.

We are looking for a solution that puts the user first.
We want to be able to give a single user, the opportunity to run complicated operations right
from her workstation, while preserving the workflow out of her machine.
To help them easily install our application, and to use it with minimum learning curve.
We want to let the user to focus on her main tasks rather than distracting them with 
data and workflow management complexity.

With this introduction, we continue with requirements.

\subsection{Distributed state}
We need a distributed solution which eliminates need to have central orchestration.
We want to decrease the dependency between running programs therefore we want to have 
some sort of distributed application which distributes the operations to all other computers.
The main requirement here is distributing the state of application between all or a number of nodes, 
in such a way that we could bootstrap a new application to a certain state.
We want to look at the network of peers as one collaborative whole which has knowledge of currently running
operations along with available datasets and preserves this state among all the nodes.

\subsection{Distributed data}
We want to be able to store result of an operation on remote computers, 
i.e. distribute it on the network. 
We also need to be able to retrieve pre-existing and produced datasets from different machines.
This comes from the nature of our workflows, where we use and produce datasets frequently. 
Normally we have datasets which represent certain models i.e.
particles of a fluid or gas inside a container and we want to run multiple operations on those datasets. 
These are normally available on different computers of an institute, and if there are multiple institutes cooperating
on a topic then we would have to fetch data from remote computers. 
Therefore we need to consider distributed data management.

\subsection{Data endpoint abstraction}
We need to abstract the absolute path of required data from users. To run every operation,
we provide certain data files as input. This is part of the manual step of running an operation and
it makes it fragile. Currently users have to take care of storing input files in correct folders before
running their scripts, or they have to copy the input files from the shared network file system to the
appropriate working directory. 
This is something that we want to change. 
We want the system to manage the input data and its absolute path.

\subsection{Server agnostic}
We want to have the same user experience on every node regardless of the machine that we work with. 
It is very common that a user moves from one workstation to another one or connects to different machines using SSH. 
We want to be able to provide the requested information to the user, no matter to which machine she connects.

We could have any number of active computers in our network which are running an instance of our program
and we want to let the users to talk to each of them and be able to launch same operation set and get the same result.

It also means that if one user initiates an operation in first computer and then goes to the next one and asks
for the status of the operation, she should be able to do that as if she is working with the original computer.

\subsection{Runtime control}
While an operation is running we want to have full control over every step of it. In traditional approaches, using job
scheduling systems, this is not possible. Except basic control such as stop, resume or similar operations, users
can not control the runtime behavior of their program. In contrast we want to control all aspects of an operation 
which will let us to implement new services specially interacting with running operations as we need.

\subsection{Easy deployment}
We look for a solution that is easy to deploy onto new machines. A fully automated installation process
is required. Unnecessary dependencies should be avoided. Users in scientific fields are not professionals in 
computer science or system administration. 
Therefore installing complex software requires system administrators to come in. 
Such installations cost money because user herself is not able to finish it without outside assistance. 
Moreover it makes it non-feasible for single users or small groups to try new solutions.

\subsection{User space solution}
We need a user space solution. It means that it should be possible to deploy, install and run the software without
having full control over the machine which is supposed to run it. 
Any software which needs administrator or root access
rights is not of our concern. Therefore the software itself along all of its dependencies should be installed and should
be able to work correctly in user space, under a normal Windows or Linux user account with no special rights except the
default ones.

\subsection{OS agnostic}
We want to be able to run the solution on both Linux machines and Windows machines. It is common that users have
two machines, one for office tasks with Windows OS and the other for running simulations with Linux. However Linux
is the favorite machine to run the software but it could be possible to run it on Windows as well.

\subsection{Light weight}
We look for a light weight solution which could be run on both laptops and PC machines, with minimum dependencies
and requirements.

%\subsection{Workflow}
%In contrast to data we are interested in workflow. We want to find a reliable approach to access and update state of our workflow on any arbitrary node which is part of our collaborative network.
