\chapter{Requirements}
\label{cha:requirements}

\section{Identical Instances}
We assume that on each machine of the network the same instance of our program is running which 
is capable of all of our operations
including A, B and C. The only consideration is the availability of 
Datasets, they are not available on all machines.

Assuming that operations A and B will run on the machines which
contain the required data, a number of questions arise here:
\begin{enumerate}
\item On which machine operations C should run? A, B or C?
\item On How to transfer the required data to that machine in an 
optimized way?
\end{enumerate}

\section{Assumptions}
During this work we have a number of assumptions. We have a certain problem which we want to focus
on rather than reintroducing solutions that already exist. For this reason we discuss regarding our 
needs.

\subsection{Collaborating Network}
We assume there is a network of computers which are available to run the tasks, each node is running an instance
of the application. We will propose our collaboration and data transfer algorithm between them later.

\subsection{Data Characteristics}
We need to discuss more about the data. In our scientific context data is mostly numerical and explains characteristics
of physical particles such as atoms and molecules. These data is being used to simulate collections of particles called
models. Although our work is not dependent on these, they help us to understand the definition of the data that
we often refer to in this report. One important aspect of the data that we are interested in is that it is not critical 
and we can reproduce it. 

\subsection{Data Transfer}
We assume a data transfer approach is already in place. This could be any file system which supports 
network storage. Rather than going into details of how data could be transferred more efficiently, we will
focus on finding which data to be transferred and from which computer to which destination.

\subsection{Workflow}
In contrast to data we are interested in workflow. We want to find a reliable approach to access and update 
state of our workflow on any arbitrary node which is part of our collaborative network.