% $Id: INF_Poster_example.tex 7714 2011-08-31 17:34:46Z tkren $
%
% TU Wien - Faculty of Informatics
% poster template
%
% This template is using the beamer document class and beamerposter package, see
% <http://www.ctan.org/tex-archive/macros/latex/contrib/beamer/>
% <http://www.ctan.org/tex-archive/macros/latex/contrib/beamerposter/>
% <http://www-i6.informatik.rwth-aachen.de/~dreuw/latexbeamerposter.php>
%
% For questions and comments send an email to
% Thomas Krennwallner <tkren@kr.tuwien.ac.at>
%

\documentclass[final,hyperref={pdfpagelabels=true}]{beamer}

\usepackage{TUINFPST}

\usepackage{lipsum}
 
%\title[Computational Intelligence]{Interactive Computer Generated Architecture}
% if you have a long title looking squeezed on the poster, just force
% some distance:
\title[Communication and Media Engineering]{Distributed Flow Control and Intelligent Data Transfer in High Performance Networks}
% if you have a long title looking squeezed on the poster, just force
% some distance:
% \title[Computational Intelligence]{%
%   Integration of Conjunctive Queries over \\[0.2\baselineskip]%
%   Description Logics into HEX-Programs %\\[0.2\baselineskip]%
% }
\author[msadeghi@stud.hs-offenburg.de]{Mehdi Sadeghi}
\institute[]{%
  Hochschule für Technik, Wirtschaft und Medien Offenburg\\[0.25\baselineskip]
  Fakultät Medien und Informationswesen\\[0.25\baselineskip]
  %Arbeitsbereich: Wissensbasierte Systeme\\[0.25\baselineskip]
  Professorin: Dr. Katharina Mehner-Heindl\\[0.25\baselineskip]
  Betreuer: Dr. Adham Hashibon
}
%\institute{
%Fraunhofer Institute for Mechanics of Materials IWM\\[0.25\baselineskip]
%Wöhlerstraße 11\\[0.25\baselineskip]
%79108 Freiburg \\[0.25\baselineskip]
%Betreuer: Dr. Adham Hashibon}
\titlegraphic{\includegraphics[height=52mm]{iwm}}
\date[\today]{\today}
\subject{epilog}
\keywords{my kwd1, my kwd2}

%%%%%%%%%%%%%%%%%%%%%%%%%%%%%%%%%%%%%%%%%%%%%%%%%%%%%%%%%%%%%%%%%%%%%%%%%%%%%%%%%%%%%%

% Display a grid to help align images 
%\beamertemplategridbackground[12.7mm]

% play around with the background colors
% \setbeamercolor{background canvas}{bg=yellow}

% use a background picture
% \usebackgroundtemplate{%
%   \includegraphics[width=\paperwidth]{logo_KBS_2_CMYK}
% }

% play around with block colors
\setbeamercolor{block body}{fg=black,bg=white}
\setbeamercolor{block title}{fg=TuWienBlue,bg=white}

\setbeamertemplate{block begin}{
  \begin{beamercolorbox}{block title}%
    \begin{tikzpicture}%
      \node[draw,rectangle,line width=3pt,rounded corners=0pt,inner sep=0pt]{%
        \begin{minipage}[c][2cm]{\linewidth}
          \centering\textbf{\insertblocktitle}
        \end{minipage}
      };
    \end{tikzpicture}%
  \end{beamercolorbox}
  \vspace*{1cm}
  \begin{beamercolorbox}{block body}%
}

\setbeamertemplate{block end}{
  \end{beamercolorbox}
  \vspace{2cm}
}

% setup postit
\setbeamercolor{postit}{fg=black,bg=yellow} 
\newenvironment{postit}
{\begin{beamercolorbox}[sep=1em,wd=7cm]{postit}}
{\end{beamercolorbox}}


% for crop marks, uncomment the following line
\usepackage[cross,width=88truecm,height=123truecm,center]{crop}

%%%%%%%%%%%%%%%%%%%%%%%%%%%%%%%%%%%%%%%%%%%%%%%%%%%%%%%%%%%%%%%%%%%%%%%%%%%%%%%%%%%%%%

\begin{document}

% We have a single poster frame.
\begin{frame}
  \begin{columns}[t]
    % ---------------------------------------------------------%
    % Set up a column
    \begin{column}{.45\textwidth}
      \begin{block}{Introduction}
European scientific communities launch many experiments every day, resulting in huge amounts
of data. Specifically in \textbf{molecular dynamics} and material science fields there are many different
simulation software which are being used to accomplish \textbf{multi scale modeling} tasks. These tasks
often involve running \textbf{multiple simulation programs} over the existing data or the data which is
produced by other simulation software. Depending on the amount of the data and the desired type of simulation
these tasks could take many days to finish. The runtime order of these softwares and 
their input data are normally defined in \textbf{scripts written by users} which is the simplest
form of workflow management.

Such experiments are the source of many high performance computing (HPC) problems, specially data transfer
and workflow management. This thesis is an effort to know the \textbf{main data transfer scenarios} in HPC experiments 
and try to address them in a distributed manner with a collective but decentralized approach toward workflow management
and dataset storage and query.

      \end{block}
      
      \begin{block}{Requirements}
      The main ones as exported during the meetings are these:
        \begin{itemize}
        \item User has no idea where the required data is located.
        \item User should have same experience regardless of the peer to which she connects.
        \item System should have \textbf{no central brokers}.
        \item Required data are distributed on the network.
        \item Runtime control over task execution.
        \item The solution should be distributed.
        \item Easily deployable.
        \item User space solution
        \item Light weight
        \end{itemize}
      \end{block}
    \end{column}
    % ---------------------------------------------------------%
    % end the column

    % ---------------------------------------------------------%
    % Set up a column 
    \begin{column}{.45\textwidth}
      \begin{block}{Related Work}
        Many scientific groups have produced tools to facilitate workflow management, job scheduling,
        service discovery, resource management and etc. Some of these are very localized and some are
        too heavy for an agile group to launch and maintain it. The job schedulers and queue management
        systems such as Sun Grid Engine (SGE) are tailored toward job management and they do not give
        control over task executions in runtime.
        Here are a number of similar works:
        \begin{itemize}
        \item UNICORE: is a grid computing technology for resources such as supercomputers or cluster systems and information stored in databases.\cite{unicore}
        \item Distributed File Systems, centralized or decentralized
        \item Porto: The Porto platform will provide a lightweight and flexible system for data and workflow management.\cite{porto}
        \end{itemize}
      \end{block}
      
      \begin{block}{Main Scenarios}
      We categorize every possible operation into two main groups:
      \begin{enumerate}
      \item Sequential operations (or linear). The algebraic notation of this operation would be:
      \[ Operation(A + B) = Operation(A) + Operation(B) \]
      \item Parallel operations (or non-linear). This operation is not equal to previous one:
      \[ Operation(A + B) \neq Operation(A) + Operation(B) \]
      \end{enumerate}
      \end{block}
      
      \begin{block}{Proposed Solution}
      We propose to make a peer-to-peer collaborative application which will run on every resource (computer) in our network, this computer (which we call it peer) will then become one part of the network. In the beginning there is one peer network but this network can be extended to contain multiple other peers. Here are main characteristics:
      \begin{itemize}
      \item The solution is peer-to-peer
      \item Based on publish-subscribe pattern
      \item Operations are \textbf{asynchronous}
      \item API calls return an \textbf{operation unique ID}
      \item Distribute operation store will be synchronized with peers
      \item \textbf{Peers subscribe to each other}
      \item Every \textbf{pear publishes news} on every \textbf{state changes}
      \item Peer can launch operations \textbf{recursively} on itself
      \item Peers can launch operations on others
      \item Peers keep state for \textbf{Operations}, \textbf{Datasets} and other \textbf{peer list}
      \item Peers and application components are \textbf{loosely coupled}
      \end{itemize}
      
      In our design the simplest operation is not dependent on any other operation, and we know how to handle it. Other operations are made from these atomic solutions.
      
        \begin{equation}
          a^2+b^2=c^2
        \end{equation}
      \end{block}

      \begin{block}{Network Design}
        \begin{equation}
          a^2+b^2=c^2
        \end{equation}
      \end{block}

      \begin{block}{The Prototype}
        %\alert{Watch out for } \cite{ff2010}
      \end{block}
	  \begin{block}{Conclusion}
	  Even though there are many solutions designed for HPC problems, still there are requirements for agile groups which are not satisfied, such as user friendly \textit{smart} applications which hide the complexity
	  of the data query and storage from end users, less deploy and maintenance cost and more flexibility and control over the application.
	  During this work we tried to address these needs and create a light weight
	  solution based on open technologies which run in user space and allows running simple and complicated scenarios over the data. Our approach is very flexible to be extended and it is easy to build new services on top
	  of the existing framework which provides the distributed operation and storage mechanisms to applications.
	  \end{block}
      \begin{block}{References}
        % this is just an example, use BibTeX!
        \begin{thebibliography}{999}
        \bibitem[UNICORE]{unicore}
        UNICORE atrcile on Wikipedia
        \url{http://en.wikipedia.org/wiki/UNICORE}
        
        \bibitem[Porto]{porto}
        The Porto platform
        \url{https://github.com/NanoSim/Porto}
%        \bibitem[Foo~and~Fu, 2010]{ff2010}
%          Foo, B.; and Fu, B.
%          2010.
%          \newblock {On logical representations of hackerisms}.
%          {\em J.~Log.~Hack.} 1:1--2.
%          
%        \bibitem[Crock~et~al., 2010]{ck2010}
%          Crock, A; Cruft, B.; and Kludge, C.
%          2010.
%          \newblock {Decomposing junk code}.
%          Manuscript.
          
        \end{thebibliography}
      \end{block}
    \end{column}
    % ---------------------------------------------------------%
    % end the column
  \end{columns}

%  \begin{tikzpicture}[remember picture,overlay]
%    \node[inner sep=0pt,xshift=-30cm,yshift=23cm] at (current page.east) {%
%      \begin{postit}%
%        Post-It time!%
%      \end{postit}%
%    }; 
%  \end{tikzpicture}
  
\end{frame}

\end{document}

%%% Local Variables:
%%% TeX-PDF-mode: t
%%% TeX-debug-bad-boxes: t
%%% TeX-master: t
%%% TeX-parse-self: t
%%% TeX-auto-save: t
%%% reftex-plug-into-AUCTeX: t
%%% End:
