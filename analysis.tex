\chapter{Problem Analysis}
\label{cha:analysis}

There are a number of possible use cases in our problem domain. To demonstrate these cases we assume
we have a number of nodes and datasets respectively, but they are not necessarily on the same nodes.
In the following paragraphs we explain possible combinations of operations, nodes and datasets.

\section{Operations}
\subsection{Types}
In every scenario we want to run an operation which could be linear or non-linear.

\subsubsection{Linear Operation}
Being linear means that the operation
could be broken into its components and then run in parallel or series. Here is algebic notation
of a linear operation which acts on two datasets:

\[ Operation(A + B) = Operation(A) + Operation(B) \]

Being linear or non-linear only matters when we have to operate on more that one dataset.

\subsubsection{Non-linear Operation}
In contrast to linear there is non-linear operation. This means that this kind of operation has dependant parts and
those parts could not run in parallel:

\[ Operation(A + B) \neq Operation(A) + Operation(B) \]

\subsection{Input/Output}
For each operation we need one or more datasets which may be available on the same node that wants to run the operation
initially or could reside on other nodes. 

\subsubsection{Input}
Input files are normally not mission critical and could be reproduced.

\subsubsection{Output}
Operations create output datasets which normally are small in size, threfore we ignore the transfer cost of operation
results in our work.

\subsubsection{Locating}
Every input or output needs an explicit address, an endpoint, either local or remote.

\section{Dataset Identification}
When we ask for an operation and we want to store the result somewhere on the network we have to think about the result name. 
We need a consistent way of naming datasets. If we ask users to provide resulting dataset names it will break soon, we need to 
let user to somehow give some \textbf{tags} but not the real names. We have to let the user know about the result name but also
let her to look for datasets by providing some tags.

The simplest problems that will happen if we store datasets with similar names is redundant work in the network. Peers will start 
to process and override the same dataset.

\subsection{Data Manipulation}
We will neet to let users to manipulate currently existing datasets, but very fast it comes to mind that not every dataset should be
writable, we will need to categorize and identify our datasets based on some criteria. These problems are not part of my thesis but
We mention it as part of problem analysis.

%We consider three different approaches toward preparing required data for operations.
%\subsubsection{Conventional Approach}
%in this approach we put the required data on a network file system and all
%application instances will access it there. We will utilize an NFS mounted file system.
%\subsubsection{Centralized Approach}
%in this approach we will have a central instance which will orchestrate operation
%delegation and operation output forwarding to other nodes.
%\subsubsection{Decentralized Approach}
%in this approach we will eliminate the orchestrator node and the network of
%application instances should collaborate in a decentralized fashion to keep track of data and control flow for each
%task.

%For every approach we will run performance tests and we will compare the results.

%\subsubsection{Method}
%For each scenario we will analyze the possible combinations of data and operations and we will discuss how to 
%deliver the input data and where to store output data. We will discuss workflow management and data transfer further in this chapter.


% TODO: I might need to introduce "Decision Tree"
\section{Decision Making}
The main decision that we need to make at every scenario is whether we should transfer the required data or we
need to delegate the operation to an instance on a node which already has the data. To make a decision we need to
answer a number of questions. First we need to know the location of the data:

\begin{enumerate}
\item Is the data available locally?
\item If not, is the data available on another node? -- Here only the physical location of data matters not the instance
controlling it.
\end{enumerate}


\section{Scenarios}
We begin with a simple scenario and we gradually add details to it and build new scenarios.

\subsection{Scenario 1 (UC1)}
In this scenario we have a linear operation, e.g. \(Op^A\) on \(Node^A\) which
requires one single dataset such as \( Dataset^1 \) which is available on one of the other peers.

%\subsubsection{Conditions} \( Dataset^1 \) is not available on \( Node^A \) and the operation is linear.
%\subsubsection{Consequences} With these conditions we either should transfer \( Dataset^1 \)
%to local node or in case of availability delegate \(Op^A\) to the node which already has \( Dataset^1 \).

We have a distributed network of collaborating servers, where in this case, we consider two computers. 
Each server has its own storage and maintains a number of datasets on it. These servers collaborate 
together to accomplish issued commands. User in this case wants to perform one operation on a dataset
that resides only on one of the servers. 

\subsubsection{Assumptions}
There are two main assumptions here:
\begin{enumerate}
\item \textbf{The user has neither a prior knowledge where the data is stored}
\item \textbf{Nor of how many servers are present on the network}
\end{enumerate}

The user connects to one of the servers, which we call a client. This server is assumed to be part 
of the network, though it may not have any local data stores on it. The user issues, interactively
(or non-interactively) a command on a set of data providing some kind of identification. This command
is broadcast by the client to all servers in the network. All servers receive this command and check
whether they have the data locally. The server which has the data performs the operation and the others
ignore it. The result of the operation in this case, remains on the same server which the original 
dataset was on. 

\begin{itemize}
\item Note: it is assumed that at any instance of time, only one server acts as a client.
\end{itemize}

Moreover we assume the user has already queried the available data in the entire network by 
issuing something like “list datasets” which outputs dataset names and ids.

The following table shows two servers, each has one dataset. The user is connected to S2.\\

\begin{tabular}{ l c r }
\em{Server ID} & \em{ Dataset ID} & \em{ Client} \\
S1 & DS1 & No \\
S2 & DS2 & Yes \\
\end{tabular}\\

Let us assume the data sets are \(10^6\) random numbers.
Let us assume the operation is to transform the real random numbers to a set of [0 or 1 ] depending on whether the number is even or odd. 
This operation is assumed here to be a user defined method that operates on the data set.

\begin{itemize}
\item Note: A dataset can be for example defined as an object that has an id, and a one dimensional array (python list).
\end{itemize}

The user issues the command like this from a python shell: 

\begin{itemize}
\item real2bin(DS1) will result in -> Broadcast(real2bin(DS1))
\item Note: it is assumed that all functions are already defined on all servers, since they execute the same environment.
\end{itemize}

The client broadcasts this function to all servers. Each server will check if the data set with this id exists, if so will run the command. 

This means that each server, especially the client, has to “know about all data sets existing in all servers.
It does not need to have the actual data, but needs to know about it. So that when the user issues the command
above, he/she does not get a “data structure not existent” error from the client, just because the data is not
there. Hence we need some interface, or some wrapper function that checks the argument for the data type, or to 
create some proxy interface from all data to all nodes. 

\subsection{Scenario 2 (UC2)}
This is similar to scenario one, except that the operation requires two datasets to be done. We have a network of peers collaborating
to finish some linear and non-linear tasks. In this scenario we need at least three peers involved. We assume the first peer has no
data of our interest therefore it should cooperate with others to accomplish the request. Our operation in this case requires two 
different datasets which are not available on the first peer and we should access them on other peers. 

\subsubsection{Assumptions}
The main points the same:
\begin{enumerate}
\item \textbf{The user has neither a prior knowledge where the datasets are stored}
\item \textbf{Nor of how many servers are present on the network}
\item \textbf{The operation is linear}
\end{enumerate}

We assume the data distribution is like the following table:

\begin{tabular}{ l c r }
\em{Server ID} & \em{ Dataset ID} & \em{ Client} \\
S0 & --- & Yes \\
S1 & DS1 & No \\
S2 & DS2 & No \\
\end{tabular}\\

\subsubsection{Possible Approach}
In order to calculate the result we might take a number of approachers, we start with a combination of \textbf{divide and conquer} and
\textbf{produce-consume-collect} methods.

The S0, in this case, is the peer who receives the command and initiates the request. The two other peers, S1 and S2 respectively, have the required
datasets. The initiator will find the corresponding datasets and will dispatch commands to run each part on each peer and then will collect
the resulting datasets. This will be a blocking operation, we will wait until the other peers finish their parts and return the result
to us. If the output is a number it will be returned to the user, if it is a dataset it will be stored based on defined storage mechanism, 
currently we use random storage. The peer will break the operation into smaller operations each one calculating result for one of datasets, 
this \textbf{sub-operations} will be executed like \textbf{scenario 1} and the result will be collected by initiator peer.

We assume that in this case we have two arrays, each consisting of \(10^6\) random numbers. We have to first transform these datasets into
a set of [0 or 1] based on the number being even or odd (use case 1) and then we make a third dataset which contains the sum of every two
corresponding numbers in range of [0 to 2].


\begin{itemize}
\item Note: in this case each pear should be able to run the requested linear operation on one or more datasets.
\end{itemize}

The notation of above mentioned approach will be like this:

\[ Operation(A + B) = Operation(A) + Operation(B) \]

In order to run this operation in a collective way, we need to think of the type of service calls in our system, whether they are blocking or
non-blocking. Since often the operations in HPC environments are time consuming and long-running, we consider the non-blocking approach. In
this way the user will provide a dataset name for storing the result. The operation will be \textbf{submitted} to the collaborative network.
Later on user is able to query for the result using the key that she had provided at the time of submission. This allows us to design our system
in a more decentralized way, where each peer can inform others (neighbors) about a request in a \textbf{publish-subscribe} manner, where the peer
will publish a request and finish the operation. Later on the peer who has the dataset will \textbf{react} to the published request and will take
further actions, all the other peers who do not have the requirements (the dataset for now) will ignore it, however they can store the details of 
running operations for next steps, when we will come to more complex workflow.

To show more detailed version of this operation we demonstrate the steps for it:

\begin{enumerate}
\item User issues the command to S0, providing DS1, DS2 \st{and a unique name for the result}
\item System will check whether the operation is linear
\item Then it will break the command into sub-commands, each for one of datasets
\item System will generate unique ids for each sub-command
\item System will then submit the sub-commands along with dataset name and the unique id for the result dataset
to \textbf{itself}, which will cause a situation like scenario 1
\item System will next have to collect the results in a non-blocking manner which we will discuss shortly.
\end{enumerate}

\begin{itemize}
\item With the use of operation ids we eliminate the need to get a result dataset name from user but we still can accept \textbf{tags} from users.
\end{itemize}

\begin{itemize}
\item We assume every operation involving more than one dataset is made of other operations which are already defined in the system.
\end{itemize}

There is an important issue here, we create sub-operations for each operation and we run them in a non-blocking manner, this will
cause it almost impossible to return the result of operation to the user in one run. One might think that we can block and query
until the result of sub-operations are ready, but this is something that we want to avoid. Therefore to solve this issue in a 
distributed manner, we introduce an operation id for each user request. We inform all the peers via sending messages (signals) about
the new operation and it is id and sub-commands. Each peer will update this operation internally based on further received messages.
We also return the operation id to the user instead of any results. Then user will query for the result of operation, providing the 
operation id. We change the above steps like this:

\begin{enumerate}
\item User issues the command to S0, providing DS1, DS2 and a unique name for the result
\item \textbf{System will generate a unique id for the operation and will store it along with the parameters}
\item System will check whether the operation is linear
\item Then it will break the command into sub-commands, each for one of datasets
\item System will generate unique ids for each sub-command
\item \textbf{System will notify other peers about the incoming operation with related parameters}
\item System will then submit the sub-commands along with dataset name and the unique id for the result dataset
to \textbf{itself}, which will cause a situation like scenario 1
\item System will next have to collect the results in a non-blocking manner which we will discuss shortly.
\item System will return the operation id to the user
\end{enumerate}

In the other hand the other peers which are the same basically, will react to the new operation signal:

\begin{enumerate}
\item Receive operation update message
\item Make a local lookup if the operation should be added or updated
\item Add or update the operation in the local storage
\end{enumerate}

Having the operation id and local updating storage for operations we now need to find a way to collect the results.
First of all we need to decide which peer will collect the results. We take the most straigh forward for now, the 
initiator peer, which has the knowledge of existing datasets in the network along with their sizes, will pick the 
peer which contains the largest dataset as the collector peer. We explicitly decide about the collector node in the
beginning either by size or randomly amount the data container peers.

It is worthy to mention that the collector peer will then store the result based on the configured storage mechanism 
which is random storage for now, not necessarily storing on the same node.

Now we have enough information in each peer to collect, process and store the results. The peers (including the collector)
 will react to operation methods like this:

\begin{enumerate}
\item Receive operation update message
\item Make a local lookup if the operation should be added or updated
\item Add or update the operation in the local storage
\item Am I the collector? If yes do the followings:
\begin{itemize}
\item check if the sub-operations are done
\item If the sub-operations are done, collect their results
\item Process the results
\item Based on the storage mechanism store the result
\item Update the operation with the result dataset id
\item Change status of operation to "done" (we need a proper state-machine here)
\item Inform other peers about the update
\end{itemize}
\end{enumerate}

Now if user makes a query giving the operation id this would be the result:

\begin{enumerate}
\item Check operation storage
\item If the operation is marked done, return the dataset id
\item If it is not done, return the status.
\end{enumerate}
