\chapter{Proposed Solution}
\label{cha:proposal}

\section{Possible Approaches}
During this work we consider three different approaches toward preparing required data for operations.
\begin{description}
\item{Conventional Approach} in this approach we put the required data on a network file system and all
application instances will access it there utilizing NFS mounted file system or other distributed file systems.
\item{Centralized Approach} in this approach we have a central instance which will orchestrate operation
delegation and operation output forwarding to other peers.
\item{Decentralized Approach} in this approach we eliminate the orchestrator peer and the network of
application instances should collaborate in a decentralized fashion to keep track of data and control flow for each
task.
\end{description}

\section{Basic Idea}
We assume that we have the information about the Datasets
available on all of the machines i.e. in form of a distributed table
with entries containing the node address and Dataset id. Based on this
information the application can decide if it has the required data or
not. 

Based on this algorithm the initial application delegates operations to the other 
nodes (instances of the same program), where the data is available. 
Our distributed workflow manager will synchronize the information on these running operation and
will label the output data and will add it to the distributed data table.

After finishing operations A and B we will run operation C in either
of these nodes, because the required data is partially available on these
nodes. Then we have to transfer the rest of the data to one of these
nodes to run the operation C which needs both parts simultaneously.

\subsection{Break and Conquer}
\subsection{Recursive Call}
\subsection{Collectors}
\subsection{Asynchronous Calls}
\subsection{Unique IDs}


\section{Operation Types}
\subsection{Simple Operation}
\subsection{Mixed Operation}

\section{Using Prior Art}
\subsection{Data Transfer}
We can take advantage of existing Distributed File Systems
(DFS) to make the data available for operations. We can then eliminate
the complexity of data transfer between these two nodes and delegate it
to existing distributed file systems. The main point is we don't rely on
DFS for all of our decision making part but we explicitly make the 
decision which operations to run on specific nodes and then for the 
data transfer part we can use a meta or universal disk concept to deliver the
remaining data.
