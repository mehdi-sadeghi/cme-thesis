\chapter{Proposed Design}
\label{cha:proposal}

%\section{Decision Making}
%The main decision that we need to make at every scenario is whether we should transfer the required data or we
%need to delegate the operation to an instance on a node which already has the data. To make a decision we need to
%answer a number of questions. First we need to know the location of the data:

%\begin{enumerate}
%\item Is the data available locally?
%\item If not, is the data available on another node? -- Here only the physical location of data matters not the instance
%controlling it.
%\end{enumerate}

%\subsubsection{Conditions} \( Dataset^1 \) is not available on \( Node^A \) and the operation is linear.
%\subsubsection{Consequences} With these conditions we either should transfer \( Dataset^1 \)
%to local node or in case of availability delegate \(Op^A\) to the node which already has \( Dataset^1 \).

\section{Possible Approaches}
During this work we consider three different approaches toward preparing required data for operations.
\begin{description}
\item{Conventional Approach} in this approach we put the required data on a network file system and all
application instances will access it there utilizing NFS mounted file system or other distributed file systems.
\item{Centralized Approach} in this approach we have a central instance which will orchestrate operation
delegation and operation output forwarding to other peers.
\item{Decentralized Approach} in this approach we eliminate the orchestrator peer and the network of
application instances should collaborate in a decentralized fashion to keep track of data and control flow for each
task.
\end{description}

\section{Basic Idea}
We assume that we have the information about the Datasets
available on all of the machines i.e. in form of a distributed table
with entries containing the node address and Dataset id. Based on this
information the application can decide if it has the required data or
not. 

Based on this algorithm the initial application delegates operations to the other 
nodes (instances of the same program), where the data is available. 
Our distributed workflow manager will synchronize the information on these running operation and
will label the output data and will add it to the distributed data table.

After finishing operations A and B we will run operation C in either
of these nodes, because the required data is partially available on these
nodes. Then we have to transfer the rest of the data to one of these
nodes to run the operation C which needs both parts simultaneously.

\subsection{Break and Conquer}
\subsection{Recursive Call}
\subsection{Collectors}
\subsection{Asynchronous Calls}
\subsection{Unique IDs}


\section{Operation Types}
\subsection{Simple Operation}
\subsection{Mixed Operation}

\section{Using Prior Art}
\subsection{Data Transfer}
We can take advantage of existing Distributed File Systems
(DFS) to make the data available for operations. We can then eliminate
the complexity of data transfer between these two nodes and delegate it
to existing distributed file systems. The main point is we don't rely on
DFS for all of our decision making part but we explicitly make the 
decision which operations to run on specific nodes and then for the 
data transfer part we can use a meta or universal disk concept to deliver the
remaining data.


% ==============================================
% ==============================================
\section{Proposed Approach}
In order to calculate the result we might take a number of approachers, we start with a combination of \textbf{divide and conquer} and
\textbf{produce-consume-collect} methods.

The S0, in this case, is the peer who receives the command and initiates the request. The two other peers, S1 and S2 respectively, have the required
datasets. The initiator will find the corresponding datasets and will dispatch commands to run each part on each peer and then will collect
the resulting datasets. This will be a blocking operation, we will wait until the other peers finish their parts and return the result
to us. If the output is a number it will be returned to the user, if it is a dataset it will be stored based on defined storage mechanism, 
currently we use random storage. The peer will break the operation into smaller operations each one calculating result for one of datasets, 
this \textbf{sub-operations} will be executed like \textbf{scenario 1} and the result will be collected by initiator peer.

We assume that in this case we have two arrays, each consisting of \(10^6\) random numbers. We have to first transform these datasets into
a set of [0 or 1] based on the number being even or odd (use case 1) and then we make a third dataset which contains the sum of every two
corresponding numbers in range of [0 to 2].


\begin{itemize}
\item Note: in this case each pear should be able to run the requested linear operation on one or more datasets.
\end{itemize}

The notation of above mentioned approach will be like this:

\[ Operation(A + B) = Operation(A) + Operation(B) \]

In order to run this operation in a collective way, we need to think of the type of service calls in our system, whether they are blocking or
non-blocking. Since often the operations in HPC environments are time consuming and long-running, we consider the non-blocking approach. In
this way the user will provide a dataset name for storing the result. The operation will be \textbf{submitted} to the collaborative network.
Later on user is able to query for the result using the key that she had provided at the time of submission. This allows us to design our system
in a more decentralized way, where each peer can inform others (neighbors) about a request in a \textbf{publish-subscribe} manner, where the peer
will publish a request and finish the operation. Later on the peer who has the dataset will \textbf{react} to the published request and will take
further actions, all the other peers who do not have the requirements (the dataset for now) will ignore it, however they can store the details of 
running operations for next steps, when we will come to more complex workflow.

To show more detailed version of this operation we demonstrate the steps for it:

\begin{enumerate}
\item User issues the command to S0, providing DS1, DS2 \st{and a unique name for the result}
\item System will check whether the operation is linear
\item Then it will break the command into sub-commands, each for one of datasets
\item System will generate unique ids for each sub-command
\item System will then submit the sub-commands along with dataset name and the unique id for the result dataset
to \textbf{itself}, which will cause a situation like scenario 1
\item System will next have to collect the results in a non-blocking manner which we will discuss shortly.
\end{enumerate}

\begin{itemize}
\item With the use of operation ids we eliminate the need to get a result dataset name from user but we still can accept \textbf{tags} from users.
\end{itemize}

\begin{itemize}
\item We assume every operation involving more than one dataset is made of other operations which are already defined in the system.
\end{itemize}

There is an important issue here, we create sub-operations for each operation and we run them in a non-blocking manner, this will
cause it almost impossible to return the result of operation to the user in one run. One might think that we can block and query
until the result of sub-operations are ready, but this is something that we want to avoid. Therefore to solve this issue in a 
distributed manner, we introduce an operation id for each user request. We inform all the peers via sending messages (signals) about
the new operation and it is id and sub-commands. Each peer will update this operation internally based on further received messages.
We also return the operation id to the user instead of any results. Then user will query for the result of operation, providing the 
operation id. We change the above steps like this:

\begin{enumerate}
\item User issues the command to S0, providing DS1, DS2 and a unique name for the result
\item \textbf{System will generate a unique id for the operation and will store it along with the parameters}
\item System will check whether the operation is linear
\item Then it will break the command into sub-commands, each for one of datasets
\item System will generate unique ids for each sub-command
\item \textbf{System will notify other peers about the incoming operation with related parameters}
\item System will then submit the sub-commands along with dataset name and the unique id for the result dataset
to \textbf{itself}, which will cause a situation like scenario 1
\item System will next have to collect the results in a non-blocking manner which we will discuss shortly.
\item System will return the operation id to the user
\end{enumerate}

In the other hand the other peers which are the same basically, will react to the new operation signal:

\begin{enumerate}
\item Receive operation update message
\item Make a local lookup if the operation should be added or updated
\item Add or update the operation in the local storage
\end{enumerate}

Having the operation id and local updating storage for operations we now need to find a way to collect the results.
First of all we need to decide which peer will collect the results. We take the most straight forward for now, the 
initiator peer, which has the knowledge of existing datasets in the network along with their sizes, will pick the 
peer which contains the largest dataset as the collector peer. We explicitly decide about the collector node in the
beginning either by size or randomly amount the data container peers.

It is worthy to mention that the collector peer will then store the result based on the configured storage mechanism 
which is random storage for now, not necessarily storing on the same node.

Now we have enough information in each peer to collect, process and store the results. The peers (including the collector)
 will react to operation methods like this:

\begin{enumerate}
\item Receive operation update message
\item Make a local lookup if the operation should be added or updated
\item Add or update the operation in the local storage
\item Am I the collector? If yes do the followings:
\begin{itemize}
\item check if the sub-operations are done
\item If the sub-operations are done, collect their results
\item Process the results
\item Based on the storage mechanism store the result
\item Update the operation with the result dataset id
\item Change status of operation to "done" (we need a proper state-machine here)
\item Inform other peers about the update
\end{itemize}
\end{enumerate}

Now if user makes a query giving the operation id this would be the result:

\begin{enumerate}
\item Check operation storage
\item If the operation is marked done, return the dataset id
\item If it is not done, return the status.
\end{enumerate}
