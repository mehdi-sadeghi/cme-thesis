\chapter{Preface} 	% engl. Preface

This document contains my master's thesis report on \emph{Distributed Flow Control 
and Intelligent Data Transfer in High Performance Computing Networks}. 
It's purpose is to show what I have done 
during my thesis and what are my results. It contains the original proposal,
an overview of state of the art and details of my suggested solution to the given 
problem.

Moreover I need to answer to one question that I have been asked
while working on my thesis. That is why I have done my thesis on a High Performance
Computing (HPC) subject in an institue which is specilized in material science rather
than computer science? This question is reasonable and therefore I put this question 
first and I write my answer to it right in the beginning.

Nowadays in many scientific fileds there is a huge demand for computing resources.
These resources include but are not limited to multi-core computers, clusters and
super-computers.
Many knowledge based institues such as \textit{Fraunhofer IWM Freiburg} - where I have 
done my thesis - 
deal with huge amount of data generated by scientific experiments. Such research 
institues need to store, transfer and process this data within different
programs, specially for simulation purposes.

Such experiments need multiple programs
to run together and transfer data back and forth to generate reasonable output for
users in this case scientists and researchers. Therefore these institues are 
among the first ones that face many HPC related problems and also are number one
customers for many HPC products. They need efficient solutions to address their 
problems regarding data transfer and workflow management while running multiple 
programs on different computers in complex workflows. 

All these problems belong to computer science field even though they arise 
in a material science institute. To give a good example I should mention \emph{The
European Organization for Nuclear Research (CERN)} which is the birth place of many 
computer science projects such as
\emph{World Wide Web (WWW)} and \emph{Worldwide LHC Computing Grid (WLCG)} to name a
few. Because of these reasons the \emph{Fraunhofer IWM Freiburg} institute has been a perfect 
place for me to accomplish my master's thesis on an HPC topic.

This is \textbf{\today} version of my master's thesis document. It is written using 
normal text editors on two Linux machines and has been shared 
online using a Github repository. The source files of the document are available 
online at this address:

%
\begin{quote}
\url{https://github.com/mehdisadeghi/cme-thesis}
\end{quote}
%

If there are any comments and improvements regarding this document, the author really
appreciates an email to the following address:

\begin{center}%
\begin{tabular}{l}
\nolinkurl{msadeghi@stud.hs-offenburg.de} \\
Hochschule Offenburg\\
Mehdi Sadeghi
\end{tabular}
\end{center}

The last but not least is the template used for this document. To give a neat and
professional look to this document I have used a master's thesis template which is
provided by:
\begin{center}%
\begin{tabular}{l}
\nolinkurl{wilhelm.burger@fh-hagenberg.at} \\
Dr.\ Wilhelm Burger \\
FH Hagenberg -- Digitale Medien\\
Austria
\end{tabular}
\end{center} 
This template is available online at:

%
\begin{quote}
\url{http://staff.fh-hagenberg.at/burger/diplomarbeit/}
\end{quote}
%





