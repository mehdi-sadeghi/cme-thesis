\chapter{Discussions}
\label{cha:discussions}

\section{Possible Issues}
\subsection{High Load}
\subsection{Orphan Operations}
\subsection{Complexity Growth}

\subsection{Large Dataset Transfer}
To transfer large arrays over the network there are a number of considerations. Should the array be stored locally before transfer?
What if the array is so big that it does not fit into the machines memory? And how the array should be transferred?

Currently we assume the result datasets fit into memory, therefore there is only the question of how 
to transfer them over the network. To prevent unnecessary copies, we consider streams to send them to
other peers. In the demo application this is done with streaming sockets. The other peer will be notified and then it will fetch the desired dataset.

We need to develop a mechanism to consider dataset size for transfer. User defined files are normally small and we can safely transfer them but
system datasets are large and for any transfer some sort of control should happen.

\section{Future Work}
During this work we have focused on the aspects of the problem which were important in the context domain and we left aside many other small and big problems without considering them during this project. The main
reason was that we wanted to work on problems which were new and genuine
because for other aspects there are already many well-defined solutions
available, so we did not spend our time for them. Moreover one should 
consider that this project is not solely an implementation rather a 
research on finding ways to embed distributed solutions into other projects.

In the following sections we talk shortly about the topics which we have
not covered but this work can be extended to include them as well.
%
%\begin{itemize}
%\item Network Discovery
%\item Bootstrapping
%\item Heartbeat Support
%\item Recognizing Popular Data
%\item Failure Recovery
%\end{itemize}


\section{Network Discovery}
Currently the peers are configured in the beginning and there is no dynamic peer recognition. This might be done in a number of ways
such as sending broadcasts or using third party projects such as Zyre \cite{Zyre}.

\section{Bootstrapping}
With having address of only one peer we would be able to configure and a new peer and join the network. There should be a mechanism among
peers to identify joining and leaving peers. But our context is different than a peer-to-peer applications which peers join and leave 
frequently. In our case most of peers run a long time and bootstrapping is more a way to get the state of currently running workflows and
let others know about the new peer.

\section{Data Popularity}
There are algorithms developed to calculate data popularity over time and then replicate them over peers for easier access. If we want to 
move toward any type of data replication we would need to use this algorithms.

\section{Security}
There is no user management and secure communication in our initial requirements however this would be required if we want to manage user
rights or introduce limitations or simply to keep a history of activities for each user. Moreover to secure inter-peer communications 
we might use X.509 certificates. Further more since we've used ZeroMQ\cite{ZeroMQ} as underlying transport channel we can use its more advanced
security features such as Elliptic curve cryptography\cite{Curve} based on Curve25519\cite{Curve25519} to add perfect forward secrecy, 
arbitrary authentication backends and so on.

\section{Fault Tolerance}
In current work there is no failure recovery mechanism, since it was not part of the requirements. In case of a failure or exception in any 
collaborating peer not only the failed instance should be able to recover itself into a correct state, moreover the other peers should maintain
a valid state for on-going distributed workflows and keep their internal state up-to-date.