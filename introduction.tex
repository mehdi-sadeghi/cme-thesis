\chapter{Introduction}
\label{cha:introduction}
\section{Thesis Objectives}
The goal of this thesis is to minimize the amount of transferred data in a network of collaborating computers which we call them peers. This data belongs to operations which might be linear or non-linear
and might involve other operations as well. Each operation can be initiated in any participating
peer but the required data is not necessarily available on that computer even though
the operation result might be delivered back to the same peer.
We will focus on two important topics in this work. First one is data transfer problems between multiple 
computers which are doing a task collaboratively. Second one is the collaboration itself, 
i.e. how multiple computers will manage to finish the task in a distributed and decentralized environment.
To accomplish these objectives we will discuss our specific distributed workflow management and data transfer methods. 
We define our requirements regarding the above mentioned topics and we will extract the parameters which we are going
to assess other solutions with them. Then we will go through the currently available solutions and we will discuss them shortly to see
whether they are applicable to our problem domain with regard to our requirements.

\section{Terminology}
We will use a number of terms through this report. Here are the meaning for each.
\subparagraph{Node}
Refers to one computer in the network.
\subparagraph{Data}
When we refer to data we mean the output of scientific applications.
\subparagraph{Dataset}
Same as data with more emphasize on it as collection e.g. NumPy array.
\subparagraph{Application}
The demo software which has been developed to show case the proposed solution.
\subparagraph{Peer}
One instance of the network application which is in collaboration with others.
\subparagraph{Instance}
An instance of the application running on a node.
\subparagraph{Operation}
Linear or non-linear functions which users want to run on datasets.
\subparagraph{Task}
Same as the operation with more emphasize on the output rather than the functionality.
\subparagraph{Service}
A scientific operation being provided by the application which could be called remotely.
\subparagraph{System}
The combination of nodes, data, application, instances, operations and services as a whole.
\subparagraph{User}
A scientist, researcher or student who uses the system to run a task.

\section{Problem Context}
Whole this work is an effort to address issues of a scientific environment. Some particular characteristics
are running multiple scientific programs on different computers which need to exchange data in order to
accomplish one operation. Another task which is often done is visualization. Visualizing the operation results
,depending on the requested visualization, might require heavy computational tasks i.e. average or comparison
on data which might not be available on the same machine or might be residing partially on different computers.
The produced data often exceeds 1 GB in many experiments and it should be moved back and forth every few minutes,
therefore it is cheaper to transfer the operation rather than the data.

The problem here is not about distributing the stored data rather exchanging it between instances of the application 
talking together in runtime while doing one global task and keeping this workflow distributed. In this terms each
application instance takes care of its own data and provides a set of services. Some operations require data from
another node, therefore we have to transfer the data or run the operation on the node which contains the data. There
are a number of scenarios which we will discuss.

\section{Assumptions}
During this work we have a number of assumptions. We have a certain problem which we want to focus
on rather than reintroducing solutions that already exist. For this reason we discuss regarding our 
needs.

\subsection{Collaborating Network}
We assume there is a network of computers which are available to run the tasks, each node is running an instance
of the application. We will propose our collaboration and data transfer algorithm between them later.

\subsection{Data Characteristics}
We need to discuss more about the data. In our scientific context data is mostly numerical and explains characteristics
of physical particles such as atoms and molecules. These data is being used to simulate collections of particles called
models. Although our work is not dependent on these, they help us to understand the definition of the data that
we often refer to in this report. One important aspect of the data that we are interested in is that it is not critical 
and we can reproduce it. 

\subsection{Data Transfer}
We assume a data transfer approach is already in place. This could be any file system which supports 
network storage. Rather than going into details of how data could be transferred more efficiently, we will
focus on finding which data to be transferred and from which computer to which destination.

\subsection{Workflow}
In contrast to data we are interested in workflow. We want to find a reliable approach to access and update 
state of our workflow on any arbitrary node which is part of our collaborative network.


% TODO: I think we need to move the following parts into a new chapter called "Problem Analysis" or similar.
% TODO: Afterwards we can add another chapter called "Proposed Solution" to put the design of the solution along with detailed algorithm about it.

%\section{Analysis}
%\subsection{Actors}
%There are two types of actors in our problem domain.
%\subparagraph{User}
%A user who launches, control and monitor an operation.
%\subparagraph{Instance}
%Every instance can launch and observe an operation on other instances on other nodes.

\subsubsection{Approaches}
During this work we consider three different approaches toward preparing required data for operations.
\subparagraph{Conventional Approach} in this approach we put the required data on a network file system and all
application instances will access it there. We will utilize an NFS mounted file system.
\subparagraph{Centralized Approach} in this approach we will have a central instance which will orchestrate operation
delegation and operation output forwarding to other nodes.
\subparagraph{Decentralized Approach} in this approach we will eliminate the orchestrator node and the network of
application instances should collaborate in a decentralized fashion to keep track of data and control flow for each
task.

For every approach we will run performance tests and we will compare the results.

\subsubsection{Method}
We will discuss scenarios in \ref{cha:scenarios}. For each scenario we will analyze the possible combinations 
of data and operations and we will discuss how to 
deliver the input data and where to store output data. We will discuss workflow management in chapter 
\ref{cha:workflow} and data transfer in \ref{cha:data}.


%\section{Structure}
%This document is structured in three main conceptual parts.
%\subparagraph{Problem} In this part we will introduce the problem domain, the 
%requirements and the main problem itself.
%\subparagraph{Prior Art} We will discuss prior art in two parts. First is about
%distributed data transfer and data access technologies. The second part is about
%distributed workflow management.
%\subparagraph{Idea} In this part we will focus on our approach to address the
%problems which have been discussed in previous chapters.