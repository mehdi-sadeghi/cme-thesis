\chapter{Introduction}
\label{cha:introduction}
\section{Objectives}
We will focus on two important topics in this work. First one is data transfer problems between multiple 
computers which are doing a task collaboratively. Second one is the collaboration itself, 
i.e. how multiple computers will accomplish a collaborative task in a distributed and decentralized environment.
To accomplish these objectives we will discuss our specific distributed workflow management and data transfer
scenarios. 
We define our requirments regarding the above mentioned topics and we will extract the parameters 
which we are going
to assess other solutions with them. Then we will go through the currently available solutions.
Afterwards we will discuss their applicability according to our requirements and whether they answer 
our needs or not.
The goal of this thesis is to minimize the amount of transferred data in a network of 
collaborating computers. This data belongs to operations which might be linear or not
and require multiple steps. The operation can be initiated in any participating
computer but the required data is not necessarily available on that computer, even though
the operation result should be delivered back to the initializer computer.

\section{Terminology}
We will use a number of terms through this report. Here are the meaning for each.
\subparagraph{Node}
Each node refers to one computer in the network.
\subparagraph{Data}
When we refer to data we mean the output of scientific applications, such as NumPy types.
\subparagraph{Dataset}
Same as data with more emphasize on it as collection of e.g. Numpy types.
\subparagraph{Application}
Refers to the demo application which has been developed to show case the proposed solution.
\subparagraph{Instance}
Refers to an instance of the same application running on a node.
\subparagraph{Operation}
Any sort of operation, linear or non-linear which is being provided by the appliction.
\subparagraph{Task}
Same as the operation with more emphasize on the output rather than the functionality.
\subparagraph{Service}
A scientific operation being provided by the application which could be called remotely.
\subparagraph{System}
The combination of nodes, data, application, instances, operations and services as a
whole.
\subparagraph{User}
A scientist, researcher or student who uses the system to run a task.

\section{Problem Context}
Whole this work is an effort to address issues of a scientific environment. Some particular characteristics
are running multiple scientific programs on different computers which need to exchange data in order to
accomplish one operation. Another task which is often done is visualization. Visualizing the operation results
,depending on the requested visualization, might require heavy computational tasks i.e. avarage or comparision
on data which might not be available on the same machine or might be residing partially on different computers.
The produced data often exceeds 1GB in many experiments and it should be moved back and forth every few minutes,
therefore it is cheaper to transfer the operation rather than the data.

The problem here is not about distributing the stored data rather, data exchange between instances of the application 
talking together in runtime while doing one global task and keeping this workflow distributed. In this terms each
application instance takes care of its own data and provides a set of services. Some operations require data from
another node, therefore we have to transfer the data or run the operation on the node which contains the data. There
are a number of scenarios which we will discuss.

\section{Assumptions}
During this work we have a number of assumptions. We have a certain problem which we want to focus
on rather than reintroducing solutions that already exist. For this reason we discuss regarding our 
needs.

\subsection{Network}
We assume there is a network of computers which are availabe to run the tasks, each node is running an instance
of the application. We will propose our collaboaration and data transfer algorith between them later.

\subsection{Data Transfer}
We assume a data transfer approach is already in place. This could be any file system which supports 
network storages. Rather than going into details of how data could be transferred more efficiently, we will
focus on finding which data to be transferred and from which computer to which destination.

\section{Analysis}

\subsection{Actors}
There are two types of actors in our problem domain.
\subparagraph{User}
A user who launches, control and monitor an operation.
\subparagraph{Instance}
Every instance can launch and observe an operation on other instances on other nodes.

\subsection{Scenarios}
There are a number of possible use cases according to the desired operations. To demonstrate these cases we assume
we have four nodes, A, B, C and D and four datasets respectively. In the following paragraphs we explain possible
combinations of operations, nodes and datasets. In every scenario we want to run an operation which could be
linear or non-linear and we need data for that operation which could be on the same node that runs the operation
initially or could reside on many other nodes. Moreover there is output dataset which should be stored.

\subsubsection{Scenario 1}
In this scenario we have a central node and a star topology. We want to run a single operation which only needs 
\( dataset^A \) on \( node^A \).


%\section{Structure}
%This document is structured in three main conceptual parts.
%\subparagraph{Problem} In this part we will introduce the problem domain, the 
%requirements and the main problem itself.
%\subparagraph{Prior Art} We will discuss prior art in two parts. First is about
%distributed data transfer and data access technologies. The second part is about
%distributed workflow management.
%\subparagraph{Idea} In this part we will focus on our approach to address the
%problems which have been discussed in previous chapters.